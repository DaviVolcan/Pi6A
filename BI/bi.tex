\documentclass{article}
\usepackage{amsmath}

\begin{document}

Queremos encontrar a raiz de \( f(x) = x - \cos(x) \), ou seja, resolver \(x - \cos(x) = 0\).

\section*{Escolha do intervalo inicial}

Precisamos de um intervalo \([a, b]\) tal que \(f(a)\) e \(f(b)\) tenham sinais opostos.

- \(f(0) = 0 - \cos(0) = -1\)
- \(f(1) = 1 - \cos(1) \approx 0.4597\)

Como \(f(0) \times f(1) < 0\), sabemos que a raiz está no intervalo \([0, 1]\).

\section*{Iterações}

\textbf{1ª iteração:}
\[
c = \frac{0 + 1}{2} = 0.5
\]
\[
f(0.5) = 0.5 - \cos(0.5) \approx -0.3776
\]
Novo intervalo: \([0.5, 1]\)

\textbf{2ª iteração:}
\[
c = \frac{0.5 + 1}{2} = 0.75
\]
\[
f(0.75) = 0.75 - \cos(0.75) \approx 0.0183
\]
Novo intervalo: \([0.5, 0.75]\)

\textbf{3ª iteração:}
\[
c = \frac{0.5 + 0.75}{2} = 0.625
\]
\[
f(0.625) = 0.625 - \cos(0.625) \approx -0.18596
\]
Novo intervalo: \([0.625, 0.75]\)

\textbf{4ª iteração:}
\[
c = \frac{0.625 + 0.75}{2} = 0.6875
\]
\[
f(0.6875) = 0.6875 - \cos(0.6875) \approx -0.0845
\]
Novo intervalo: \([0.6875, 0.75]\)

\textbf{5ª iteração:}
\[
c = \frac{0.6875 + 0.75}{2} = 0.71875
\]
\[
f(0.71875) = 0.71875 - \cos(0.71875) \approx -0.03365
\]
Novo intervalo: \([0.71875, 0.75]\)

\textbf{6ª iteração:}
\[
c = \frac{0.71875 + 0.75}{2} = 0.734375
\]
\[
f(0.734375) = 0.734375 - \cos(0.734375) \approx -0.00533
\]
Novo intervalo: \([0.734375, 0.75]\)

\textbf{7ª iteração:}
\[
c = \frac{0.734375 + 0.75}{2} = 0.7421875
\]
\[
f(0.7421875) = 0.7421875 - \cos(0.7421875) \approx 0.00629
\]
Novo intervalo: \([0.734375, 0.7421875]\)

\textbf{8ª iteração:}
\[
c = \frac{0.734375 + 0.7421875}{2} = 0.73828125
\]
\[
f(0.73828125) = 0.73828125 - \cos(0.73828125) \approx 0.00078
\]

\section*{Conclusão}

A raiz aproximada de \(f(x) = 0\) é \(x \approx 0.73828\), com uma precisão de \(10^{-4}\).

\end{document}
