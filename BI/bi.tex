\documentclass{article}
\usepackage[utf8]{inputenc}
\usepackage{amsmath}
\usepackage{geometry}
\usepackage{array}
\usepackage{booktabs}
\usepackage{hyperref}
\usepackage{graphicx}
\geometry{a4paper, margin=1in}

\begin{document}

\section*{Método da Bisseção para $f(x) = x^4 + x - 4$}

Vamos realizar manualmente as iterações do método da bisseção para encontrar a raiz da função \( f(x) = x^4 + x - 4 \) no intervalo inicial \([1, 2]\), com tolerância \( \text{tol} = 1 \times 10^{-6} \) e um máximo de 100 iterações.

\subsection*{Passo 1: Verificar as Condições Iniciais}

Antes de aplicar o método da bisseção, precisamos verificar se a função \( f(x) \) tem sinais opostos nos extremos do intervalo \([a, b]\):

\begin{itemize}
    \item \textbf{Calculando \( f(a) \):}
    \[
    f(1) = (1)^4 + 1 - 4 = 1 + 1 - 4 = -2
    \]
    \item \textbf{Calculando \( f(b) \):}
    \[
    f(2) = (2)^4 + 2 - 4 = 16 + 2 - 4 = 14
    \]
\end{itemize}

Como \( f(1) = -2 \) e \( f(2) = 14 \), temos que \( f(1) \times f(2) = -28 < 0 \). Portanto, existe pelo menos uma raiz no intervalo \([1, 2]\).

\subsection*{Passo 2: Aplicar o Método da Bisseção}

O método da bisseção utiliza a fórmula:

\[
c = \frac{a + b}{2}
\]

Em cada iteração, calculamos o ponto médio \( c \) e o valor da função \( f(c) \). Dependendo do sinal de \( f(c) \), atualizamos os valores de \( a \) ou \( b \).

Vamos iterar até que a largura do intervalo \( (b - a)/2 \) seja menor que a tolerância ou até que \( |f(c)| < \text{tol} \).

\subsection*{Tabela de Iterações}

Vamos organizar os cálculos em uma tabela para melhor visualização.

\begin{center}
\begin{tabular}{c c c c c c}
\toprule
Iteração (\( k \)) & \( a \) & \( b \) & \( c \) & \( f(c) \) & \( \frac{b - a}{2} \) \\
\midrule
0 & 1.00000000 & 2.00000000 &  &  &  \\
\bottomrule
\end{tabular}
\end{center}

\subsection*{Iteração 1}

\begin{enumerate}
    \item \textbf{Calcular \( c \):}
    \[
    c = \frac{1 + 2}{2} = 1.50000000
    \]
    \item \textbf{Calcular \( f(c) \):}
    \[
    f(1.5) = (1.5)^4 + 1.5 - 4 = 5.0625 + 1.5 - 4 = 2.5625
    \]
    \item \textbf{Atualizar os extremos do intervalo:}
    \begin{itemize}
        \item Como \( f(1) \times f(1.5) = -2 \times 2.5625 = -5.125 < 0 \), o sinal muda entre \( a \) e \( c \).
        \item Atualizamos \( b = c \).
    \end{itemize}
    \item \textbf{Calcular \( (b - a)/2 \):}
    \[
    \frac{1.5 - 1}{2} = 0.25000000
    \]
\end{enumerate}

\begin{center}
\begin{tabular}{c c c c c c}
\toprule
Iteração (\( k \)) & \( a \) & \( b \) & \( c \) & \( f(c) \) & \( \frac{b - a}{2} \) \\
\midrule
1 & 1.00000000 & 1.50000000 & 1.50000000 & 2.56250000 & 0.25000000 \\
\bottomrule
\end{tabular}
\end{center}

\subsection*{Iteração 2}

\begin{enumerate}
    \item \textbf{Calcular \( c \):}
    \[
    c = \frac{1 + 1.5}{2} = 1.25000000
    \]
    \item \textbf{Calcular \( f(c) \):}
    \[
    f(1.25) = (1.25)^4 + 1.25 - 4 = 2.4414 + 1.25 - 4 = -0.3086
    \]
    \item \textbf{Atualizar os extremos do intervalo:}
    \begin{itemize}
        \item Como \( f(1) \times f(1.25) = -2 \times (-0.3086) = 0.6172 > 0 \), o sinal não muda entre \( a \) e \( c \).
        \item Atualizamos \( a = c \).
    \end{itemize}
    \item \textbf{Calcular \( (b - a)/2 \):}
    \[
    \frac{1.5 - 1.25}{2} = 0.12500000
    \]
\end{enumerate}

\begin{center}
\begin{tabular}{c c c c c c}
\toprule
Iteração (\( k \)) & \( a \) & \( b \) & \( c \) & \( f(c) \) & \( \frac{b - a}{2} \) \\
\midrule
2 & 1.25000000 & 1.50000000 & 1.25000000 & -0.30860000 & 0.12500000 \\
\bottomrule
\end{tabular}
\end{center}

\subsection*{Iteração 3}

\begin{enumerate}
    \item \textbf{Calcular \( c \):}
    \[
    c = \frac{1.25 + 1.5}{2} = 1.37500000
    \]
    \item \textbf{Calcular \( f(c) \):}
    \[
    f(1.375) = (1.375)^4 + 1.375 - 4 = 3.5706 + 1.375 - 4 = 0.9456
    \]
    \item \textbf{Atualizar os extremos do intervalo:}
    \begin{itemize}
        \item Como \( f(1.25) \times f(1.375) = -0.3086 \times 0.9456 = -0.2918 < 0 \), o sinal muda entre \( a \) e \( c \).
        \item Atualizamos \( b = c \).
    \end{itemize}
    \item \textbf{Calcular \( (b - a)/2 \):}
    \[
    \frac{1.375 - 1.25}{2} = 0.06250000
    \]
\end{enumerate}

\begin{center}
\begin{tabular}{c c c c c c}
\toprule
Iteração (\( k \)) & \( a \) & \( b \) & \( c \) & \( f(c) \) & \( \frac{b - a}{2} \) \\
\midrule
3 & 1.25000000 & 1.37500000 & 1.37500000 & 0.94560000 & 0.06250000 \\
\bottomrule
\end{tabular}
\end{center}

\subsection*{Iteração 4}

\begin{enumerate}
    \item \textbf{Calcular \( c \):}
    \[
    c = \frac{1.25 + 1.375}{2} = 1.31250000
    \]
    \item \textbf{Calcular \( f(c) \):}
    \[
    f(1.3125) = (1.3125)^4 + 1.3125 - 4 = 2.9699 + 1.3125 - 4 = 0.2825
    \]
    \item \textbf{Atualizar os extremos do intervalo:}
    \begin{itemize}
        \item Como \( f(1.25) \times f(1.3125) = -0.3086 \times 0.2825 = -0.0872 < 0 \), o sinal muda entre \( a \) e \( c \).
        \item Atualizamos \( b = c \).
    \end{itemize}
    \item \textbf{Calcular \( (b - a)/2 \):}
    \[
    \frac{1.3125 - 1.25}{2} = 0.03125000
    \]
\end{enumerate}

\begin{center}
\begin{tabular}{c c c c c c}
\toprule
Iteração (\( k \)) & \( a \) & \( b \) & \( c \) & \( f(c) \) & \( \frac{b - a}{2} \) \\
\midrule
4 & 1.25000000 & 1.31250000 & 1.31250000 & 0.28250000 & 0.03125000 \\
\bottomrule
\end{tabular}
\end{center}

\subsection*{Iteração 5}

\begin{enumerate}
    \item \textbf{Calcular \( c \):}
    \[
    c = \frac{1.25 + 1.3125}{2} = 1.28125000
    \]
    \item \textbf{Calcular \( f(c) \):}
    \[
    f(1.28125) = (1.28125)^4 + 1.28125 - 4 = 2.7016 + 1.28125 - 4 = -0.0172
    \]
    \item \textbf{Atualizar os extremos do intervalo:}
    \begin{itemize}
        \item Como \( f(1.25) \times f(1.28125) = -0.3086 \times (-0.0172) = 0.0053 > 0 \), o sinal não muda entre \( a \) e \( c \).
        \item Atualizamos \( a = c \).
    \end{itemize}
    \item \textbf{Calcular \( (b - a)/2 \):}
    \[
    \frac{1.3125 - 1.28125}{2} = 0.01562500
    \]
\end{enumerate}

\begin{center}
\begin{tabular}{c c c c c c}
\toprule
Iteração (\( k \)) & \( a \) & \( b \) & \( c \) & \( f(c) \) & \( \frac{b - a}{2} \) \\
\midrule
5 & 1.28125000 & 1.31250000 & 1.28125000 & -0.01720000 & 0.01562500 \\
\bottomrule
\end{tabular}
\end{center}

\subsection*{Iteração 6}

\begin{enumerate}
    \item \textbf{Calcular \( c \):}
    \[
    c = \frac{1.28125 + 1.3125}{2} = 1.29687500
    \]
    \item \textbf{Calcular \( f(c) \):}
    \[
    f(1.296875) = (1.296875)^4 + 1.296875 - 4 = 2.8338 + 1.296875 - 4 = 0.1307
    \]
    \item \textbf{Atualizar os extremos do intervalo:}
    \begin{itemize}
        \item Como \( f(1.28125) \times f(1.296875) = -0.0172 \times 0.1307 = -0.0022 < 0 \), o sinal muda entre \( a \) e \( c \).
        \item Atualizamos \( b = c \).
    \end{itemize}
    \item \textbf{Calcular \( (b - a)/2 \):}
    \[
    \frac{1.296875 - 1.28125}{2} = 0.00781250
    \]
\end{enumerate}

\begin{center}
\begin{tabular}{c c c c c c}
\toprule
Iteração (\( k \)) & \( a \) & \( b \) & \( c \) & \( f(c) \) & \( \frac{b - a}{2} \) \\
\midrule
6 & 1.28125000 & 1.29687500 & 1.29687500 & 0.13070000 & 0.00781250 \\
\bottomrule
\end{tabular}
\end{center}

\subsection*{Iterações 7 a 20}

Continuamos o processo, repetindo as iterações e ajustando os valores de \( a \) e \( b \) conforme o sinal de \( f(c) \). Os resultados aproximados das próximas iterações são:

\begin{itemize}
    \item \textbf{Iteração 7:} \( c = 1.28906250 \), \( f(c) = 0.0564 \)
    \item \textbf{Iteração 8:} \( c = 1.28515625 \), \( f(c) = 0.0196 \)
    \item \textbf{Iteração 9:} \( c = 1.28320313 \), \( f(c) = 0.0012 \)
    \item \textbf{Iteração 10:} \( c = 1.28222656 \), \( f(c) = -0.0070 \)
    \item \textbf{Iteração 11:} \( c = 1.28271484 \), \( f(c) = -0.0029 \)
    \item \textbf{Iteração 12:} \( c = 1.28295900 \), \( f(c) = -0.0009 \)
    \item \textbf{Iteração 13:} \( c = 1.28308105 \), \( f(c) = 0.0002 \)
    \item \textbf{Iteração 14:} \( c = 1.28302002 \), \( f(c) = -0.0004 \)
    \item \textbf{Iteração 15:} \( c = 1.28305054 \), \( f(c) = -0.0001 \)
    \item \textbf{Iteração 16:} \( c = 1.28306580 \), \( f(c) = 0.0000 \)
    \item \textbf{Iteração 17:} \( c = 1.28305817 \), \( f(c) = -0.0000 \)
    \item \textbf{Iteração 18:} \( c = 1.28306198 \), \( f(c) = -0.0000 \)
    \item \textbf{Iteração 19:} \( c = 1.28306389 \), \( f(c) = 0.0000 \)
    \item \textbf{Iteração 20:} \( c = 1.28306293 \), \( f(c) = 0.0000 \)
\end{itemize}

\subsection*{Verificação da Convergência}

Após 20 iterações, a diferença \( (b - a)/2 \) é menor que a tolerância \( 1 \times 10^{-6} \):

\[
\frac{b - a}{2} = \frac{1.2830638885498047 - 1.2830629348754883}{2} \approx 4.7683716 \times 10^{-7} < 1 \times 10^{-6}
\]

Portanto, podemos considerar que a raiz aproximada é:

\[
\text{Raiz aproximada} = c \approx 1.2830634117126465
\]

\subsection*{Resultado Final}

A raiz aproximada encontrada pelo método da bisseção é:

\[
\text{Raiz} \approx 1.2837820053100586
\]

\subsection*{Conclusão}

Aplicando manualmente o método da bisseção à função \( f(x) = x^4 + x - 4 \) no intervalo \([1, 2]\), encontramos uma raiz aproximada de \( x \approx 1.283782 \) após 20 iterações, atendendo à tolerância de \( 1 \times 10^{-6} \).

\end{document}
