\documentclass{article}
\usepackage{amsmath}
\usepackage{geometry}
\usepackage{verbatim}
\usepackage{hyperref}
\geometry{a4paper, margin=1in}

\begin{document}

\section*{Método de Newton-Raphson para $f(x) = x^4 + x - 4$}

Vamos aplicar o método de Newton-Raphson passo a passo para encontrar a raiz da função \( f(x) = x^4 + x - 4 \), com chute inicial \( x_0 = 1{,}5 \), tolerância \( \text{tol} = 1 \times 10^{-6} \) e máximo de 100 iterações.

\subsection*{Passo 1: Definir as funções}

\begin{itemize}
    \item \textbf{Função original}: \( f(x) = x^4 + x - 4 \)
    \item \textbf{Derivada da função}: \( f'(x) = 4x^3 + 1 \)
\end{itemize}

\subsection*{Passo 2: Iteração do Método de Newton-Raphson}

O método de Newton-Raphson utiliza a fórmula:

\[
x_{n+1} = x_n - \frac{f(x_n)}{f'(x_n)}
\]

Vamos calcular \( x_1, x_2, x_3, \ldots \) até que \( |x_{n+1} - x_n| < \text{tol} \).

\subsubsection*{Iteração 0}

\textbf{Chute inicial}: \( x_0 = 1{,}5 \)

1. \textbf{Calcular \( f(x_0) \)}:

\[
f(1{,}5) = (1{,}5)^4 + 1{,}5 - 4 = 5{,}0625 + 1{,}5 - 4 = 2{,}5625
\]

2. \textbf{Calcular \( f'(x_0) \)}:

\[
f'(1{,}5) = 4(1{,}5)^3 + 1 = 4(3{,}375) + 1 = 13{,}5 + 1 = 14{,}5
\]

3. \textbf{Atualizar \( x_1 \)}:

\[
x_1 = x_0 - \frac{f(x_0)}{f'(x_0)} = 1{,}5 - \frac{2{,}5625}{14{,}5} \approx 1{,}5 - 0{,}1767241 = 1{,}3232759
\]

\subsubsection*{Iteração 1}

\textbf{Novo valor}: \( x_1 = 1{,}3232759 \)

1. \textbf{Calcular \( f(x_1) \)}:

\[
\begin{aligned}
f(1{,}3232759) &= (1{,}3232759)^4 + 1{,}3232759 - 4 \\
&= 3{,}0654 + 1{,}3232759 - 4 \\
&= 0{,}3886759
\end{aligned}
\]

2. \textbf{Calcular \( f'(x_1) \)}:

\[
\begin{aligned}
f'(1{,}3232759) &= 4(1{,}3232759)^3 + 1 \\
&= 4(2{,}31574) + 1 \\
&= 9{,}26296
\end{aligned}
\]

3. \textbf{Atualizar \( x_2 \)}:

\[
x_2 = x_1 - \frac{f(x_1)}{f'(x_1)} = 1{,}3232759 - \frac{0{,}3886759}{9{,}26296} \approx 1{,}3232759 - 0{,}0419669 = 1{,}2813090
\]

\subsubsection*{Iteração 2}

\textbf{Novo valor}: \( x_2 = 1{,}2813090 \)

1. \textbf{Calcular \( f(x_2) \)}:

\[
\begin{aligned}
f(1{,}2813090) &= (1{,}2813090)^4 + 1{,}2813090 - 4 \\
&= 2{,}695355 + 1{,}2813090 - 4 \\
&= -0{,}023336
\end{aligned}
\]

2. \textbf{Calcular \( f'(x_2) \)}:

\[
\begin{aligned}
f'(1{,}2813090) &= 4(1{,}2813090)^3 + 1 \\
&= 4(2{,}10145) + 1 \\
&= 9{,}4058
\end{aligned}
\]

3. \textbf{Atualizar \( x_3 \)}:

\[
x_3 = x_2 - \left( \frac{f(x_2)}{f'(x_2)} \right) = 1{,}2813090 - \left( \frac{-0{,}023336}{9{,}4058} \right) \approx 1{,}2813090 + 0{,}0024822 = 1{,}2837912
\]

\subsubsection*{Iteração 3}

\textbf{Novo valor}: \( x_3 = 1{,}2837912 \)

1. \textbf{Calcular \( f(x_3) \)}:

\[
\begin{aligned}
f(1{,}2837912) &= (1{,}2837912)^4 + 1{,}2837912 - 4 \\
&= 2{,}716296 + 1{,}2837912 - 4 \\
&= 0{,}000088
\end{aligned}
\]

2. \textbf{Calcular \( f'(x_3) \)}:

\[
\begin{aligned}
f'(1{,}2837912) &= 4(1{,}2837912)^3 + 1 \\
&= 4(2{,}10962) + 1 \\
&= 9{,}43848
\end{aligned}
\]

3. \textbf{Atualizar \( x_4 \)}:

\[
x_4 = x_3 - \frac{f(x_3)}{f'(x_3)} = 1{,}2837912 - \frac{0{,}000088}{9{,}43848} \approx 1{,}2837912 - 0{,}0000093 = 1{,}2837819
\]

\subsubsection*{Iteração 4}

\textbf{Novo valor}: \( x_4 = 1{,}2837819 \)

1. \textbf{Calcular \( f(x_4) \)}:

\[
f(1{,}2837819) = (1{,}2837819)^4 + 1{,}2837819 - 4 \approx 0{,}0000001
\]

2. \textbf{Calcular \( f'(x_4) \)}:

\[
f'(1{,}2837819) = 4(1{,}2837819)^3 + 1 \approx 9{,}43824
\]

3. \textbf{Atualizar \( x_5 \)}:

\[
x_5 = x_4 - \frac{f(x_4)}{f'(x_4)} = 1{,}2837819 - \frac{0{,}0000001}{9{,}43824} \approx 1{,}2837819 - 0{,}00000001 = 1{,}2837817
\]

\subsection*{Convergência}

Calculamos \( |x_5 - x_4| = |1{,}2837817 - 1{,}2837819| = 2 \times 10^{-7} \), que é menor que a tolerância \( 1 \times 10^{-6} \).

\subsection*{Resposta Final}

A raiz aproximada é \( x \approx 1{,}2837817 \).

\subsection*{Comparação com o resultado do código}

O código fornece a raiz como:

\begin{verbatim}
print(raiz)
# Saída: 1.2837816658635381
\end{verbatim}

Nosso cálculo manual está em excelente concordância com o resultado do código.

\subsection*{Conclusão}

Utilizando o método de Newton-Raphson, encontramos a raiz da função \( f(x) = x^4 + x - 4 \) como aproximadamente \( x = 1{,}2837817 \) após 5 iterações, com uma tolerância de \( 1 \times 10^{-6} \).

\end{document}
