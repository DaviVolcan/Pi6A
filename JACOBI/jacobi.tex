\documentclass{article}
\usepackage[utf8]{inputenc}
\usepackage{amsmath}
\usepackage{amssymb}
\usepackage{geometry}
\geometry{a4paper, margin=1in}
\title{oi}
\author{Volcan LAB}
\date{September 2024}

\begin{document}

\maketitle



\title{Resolução Manual do Método de Jacobi}
\author{}
\date{}

\maketitle

\section*{Introdução}

Vamos resolver manualmente os cálculos do método iterativo de Jacobi implementado no código fornecido.

\section*{Dados Iniciais}

Matriz $A$ e vetor $B$:

\[
A = \begin{bmatrix}
10 & 2 & 1 \\
1 & 5 & 1 \\
2 & 3 & 10 \\
\end{bmatrix}, \quad
B = \begin{bmatrix}
7 \\
-8 \\
6 \\
\end{bmatrix}
\]

Vetor inicial $X^{(0)}$:

\[
X^{(0)} = \begin{bmatrix}
0.7 \\
-1.6 \\
0.6 \\
\end{bmatrix}
\]

\section*{Passo 1: Cálculo das Matrizes $D$ e $I$}

Para o método de Jacobi, definimos:

\[
D_{ij} = \begin{cases}
0, & \text{se } i = j \\
-\dfrac{A_{ij}}{A_{ii}}, & \text{se } i \neq j \\
\end{cases}
\]

\[
I_i = \dfrac{B_i}{A_{ii}}
\]

\subsection*{Cálculo de $D$ e $I$}

\textbf{Para $i = 0$:}

\begin{align*}
D_{00} &= 0 \\
D_{01} &= -\dfrac{A_{01}}{A_{00}} = -\dfrac{2}{10} = -0.2 \\
D_{02} &= -\dfrac{A_{02}}{A_{00}} = -\dfrac{1}{10} = -0.1 \\
I_0 &= \dfrac{B_0}{A_{00}} = \dfrac{7}{10} = 0.7 \\
\end{align*}

\textbf{Para $i = 1$:}

\begin{align*}
D_{11} &= 0 \\
D_{10} &= -\dfrac{A_{10}}{A_{11}} = -\dfrac{1}{5} = -0.2 \\
D_{12} &= -\dfrac{A_{12}}{A_{11}} = -\dfrac{1}{5} = -0.2 \\
I_1 &= \dfrac{B_1}{A_{11}} = \dfrac{-8}{5} = -1.6 \\
\end{align*}

\textbf{Para $i = 2$:}

\begin{align*}
D_{22} &= 0 \\
D_{20} &= -\dfrac{A_{20}}{A_{22}} = -\dfrac{2}{10} = -0.2 \\
D_{21} &= -\dfrac{A_{21}}{A_{22}} = -\dfrac{3}{10} = -0.3 \\
I_2 &= \dfrac{B_2}{A_{22}} = \dfrac{6}{10} = 0.6 \\
\end{align*}

\textbf{Matriz $D$:}

\[
D = \begin{bmatrix}
0 & -0.2 & -0.1 \\
-0.2 & 0 & -0.2 \\
-0.2 & -0.3 & 0 \\
\end{bmatrix}
\]

\textbf{Vetor $I$:}

\[
I = \begin{bmatrix}
0.7 \\
-1.6 \\
0.6 \\
\end{bmatrix}
\]

\section*{Passo 2: Iterações do Método de Jacobi}

Critério de parada: erro relativo percentual menor que $\text{tol} = 5\%$ ou número máximo de iterações $\text{max\_iter} = 4$.

\subsection*{Iteração 1 ($k = 1$)}

Calcular $X^{(1)} = D \cdot X^{(0)} + I$.

\textbf{Cálculo de $Y = D \cdot X^{(0)}$:}

\begin{align*}
Y_0 &= D_{01} X^{(0)}_1 + D_{02} X^{(0)}_2 \\
&= (-0.2)(-1.6) + (-0.1)(0.6) \\
&= 0.32 - 0.06 = 0.26 \\
\\
Y_1 &= D_{10} X^{(0)}_0 + D_{12} X^{(0)}_2 \\
&= (-0.2)(0.7) + (-0.2)(0.6) \\
&= -0.14 - 0.12 = -0.26 \\
\\
Y_2 &= D_{20} X^{(0)}_0 + D_{21} X^{(0)}_1 \\
&= (-0.2)(0.7) + (-0.3)(-1.6) \\
&= -0.14 + 0.48 = 0.34 \\
\end{align*}

\textbf{Cálculo de $X^{(1)} = Y + I$:}

\[
X^{(1)} = \begin{bmatrix}
0.26 + 0.7 \\
-0.26 - 1.6 \\
0.34 + 0.6 \\
\end{bmatrix} = \begin{bmatrix}
0.96 \\
-1.86 \\
0.94 \\
\end{bmatrix}
\]

\textbf{Cálculo do erro relativo percentual:}

\[
\text{erro}_i = \left| \dfrac{X^{(1)}_i - X^{(0)}_i}{X^{(1)}_i} \right| \times 100\%
\]

\begin{align*}
\text{erro}_0 &= \left| \dfrac{0.96 - 0.7}{0.96} \right| \times 100\% \approx 27.08\% \\
\text{erro}_1 &= \left| \dfrac{-1.86 - (-1.6)}{-1.86} \right| \times 100\% \approx 13.98\% \\
\text{erro}_2 &= \left| \dfrac{0.94 - 0.6}{0.94} \right| \times 100\% \approx 36.17\% \\
\end{align*}

Como os erros não estão abaixo de $5\%$, procedemos para a próxima iteração.

\subsection*{Iteração 2 ($k = 2$)}

Atualizar $X^{(0)} = X^{(1)}$.

\textbf{Cálculo de $Y = D \cdot X^{(1)}$:}

\begin{align*}
Y_0 &= (-0.2)(-1.86) + (-0.1)(0.94) \\
&= 0.372 - 0.094 = 0.278 \\
\\
Y_1 &= (-0.2)(0.96) + (-0.2)(0.94) \\
&= -0.192 - 0.188 = -0.38 \\
\\
Y_2 &= (-0.2)(0.96) + (-0.3)(-1.86) \\
&= -0.192 + 0.558 = 0.366 \\
\end{align*}

\textbf{Cálculo de $X^{(2)} = Y + I$:}

\[
X^{(2)} = \begin{bmatrix}
0.278 + 0.7 \\
-0.38 - 1.6 \\
0.366 + 0.6 \\
\end{bmatrix} = \begin{bmatrix}
0.978 \\
-1.98 \\
0.966 \\
\end{bmatrix}
\]

\textbf{Cálculo do erro relativo percentual:}

\begin{align*}
\text{erro}_0 &= \left| \dfrac{0.978 - 0.96}{0.978} \right| \times 100\% \approx 1.84\% \\
\text{erro}_1 &= \left| \dfrac{-1.98 - (-1.86)}{-1.98} \right| \times 100\% \approx 6.06\% \\
\text{erro}_2 &= \left| \dfrac{0.966 - 0.94}{0.966} \right| \times 100\% \approx 2.69\% \\
\end{align*}

Como ainda temos um erro acima de $5\%$ no segundo elemento, continuamos.

\subsection*{Iteração 3 ($k = 3$)}

Atualizar $X^{(1)} = X^{(2)}$.

\textbf{Cálculo de $Y = D \cdot X^{(2)}$:}

\begin{align*}
Y_0 &= (-0.2)(-1.98) + (-0.1)(0.966) \\
&= 0.396 - 0.0966 = 0.2994 \\
\\
Y_1 &= (-0.2)(0.978) + (-0.2)(0.966) \\
&= -0.1956 - 0.1932 = -0.3888 \\
\\
Y_2 &= (-0.2)(0.978) + (-0.3)(-1.98) \\
&= -0.1956 + 0.594 = 0.3984 \\
\end{align*}

\textbf{Cálculo de $X^{(3)} = Y + I$:}

\[
X^{(3)} = \begin{bmatrix}
0.2994 + 0.7 \\
-0.3888 - 1.6 \\
0.3984 + 0.6 \\
\end{bmatrix} = \begin{bmatrix}
0.9994 \\
-1.9888 \\
0.9984 \\
\end{bmatrix}
\]

\textbf{Cálculo do erro relativo percentual:}

\begin{align*}
\text{erro}_0 &= \left| \dfrac{0.9994 - 0.978}{0.9994} \right| \times 100\% \approx 2.14\% \\
\text{erro}_1 &= \left| \dfrac{-1.9888 - (-1.98)}{-1.9888} \right| \times 100\% \approx 0.44\% \\
\text{erro}_2 &= \left| \dfrac{0.9984 - 0.966}{0.9984} \right| \times 100\% \approx 3.25\% \\
\end{align*}

Agora, todos os erros estão abaixo de $5\%$. Portanto, podemos interromper as iterações.

\section*{Resultado Final}

O vetor $X$ aproximado após as iterações é:

\[
X = \begin{bmatrix}
0.978 \\
-1.98 \\
0.966 \\
\end{bmatrix}
\]

Este é o resultado obtido no código após as iterações com o critério de parada definido.

\end{document}

\section{Introduction}

\end{document}
